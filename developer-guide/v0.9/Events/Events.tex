%%%%%%%%%%%%%%%%%%%%%%%%%%%%%%%%%%%%%%%%%%%%%%%%%%%%%%%%%%%%%%%%%%%%%%%%%%%%%%%%%%%%%%%%%%%%%%%%%%%%%%%%%%%%%%%%%%%%%%%%
\newpage
\chapter {\Large{Event Notifiers}}

Siminov Framework provides few event notifiers which gets triggered based on particular action. Application have to provide implementation for these event notifiers and register them with Siminov.

	\begin{center}
		\colorbox{grey}{
			\parbox[t]{.8\linewidth}{
			\fontsize{11pt}{11pt}\selectfont % The first argument for fontsize is the font size of the text and the second is the line spacing - you may need to play with these for your particular title
			\vspace*{0.1cm} % Space between the start of the title and the top of the grey box
		
			\hfill \textbf{Note} \\
				
				\begin{enumerate}

					\item \small If you use Siminov Hybrid Framework it is mandatory to register for Events.

					\item \small If you use only Siminov Native ORM Framework it is not mandatory to register for events.

					\item \small Event Handler can be define in both Native and Web.

					\item \small Native Event Handler: If you want to handle Event in Native then define Event Handler using Java. Specify full Java class path and name in ApplicationDescriptor.si.xml.

					\item \small Web Event Handler: If you want to handle Event in Web then define Event Handler using JavaScript. Specify only JavaScript Function name in ApplicationDescriptor.si.xml.

					\item \small Both (Native/Web) Event Handler: If you want to handle Event in both Native and Web then define Event Handler in both Java and JavaScript. Specify full Java class path and name in ApplicationDescriptor.si.xml, no need to define for JavaScript because Siminov will automatically accume same name for it.
	
				\end{enumerate}
				
			\vspace*{0.0cm} % Space between the end of the title and the bottom of the grey box
			}
		}

	\end{center}





\section{ISiminov Events} It provide API's related to life cycle of Siminov Framework

		\par
		\textbf{Example:} ISiminov Events Notifier
			\lstinputlisting[language=Java]{Resources/isiminov_events_notifier.txt}


		\begin{enumerate}

			\item \small \textbf{First Time Siminov Initialized - firstTimSiminovInitialized()}: It is triggered when Siminov is initialized for first time. In this you can perform tasks which are related to initialization of things only first time of application starts.

					\par
					\textbf{Example:}
	
					\par
					 Preparing initial data for application, which is required by application in its life time, Since it is to be done only once, therefore we will use firstTimeSiminovInitialized API.
					\lstinputlisting[language=Java]{Resources/siminov_hybrid_template_application_first_time_siminov_initialized_event_notifier_example.txt}	
		
			
					\begin{center}
						\colorbox{grey}{
						\parbox[t]{.8\linewidth}{
							\fontsize{11pt}{11pt}\selectfont % The first argument for fontsize is the font size of the text and the second is the line spacing - you may need to play with these for your particular title
							\vspace*{0.1cm} % Space between the start of the title and the top of the grey box
		
							\hfill \textbf{Note} \\
							This API will be triggered only once when Siminov is initialized first time.
				
							\vspace*{0.0cm} % Space between the end of the title and the bottom of the grey box
						}
					}


					\end{center}


			\item \small \textbf{Siminov Initialized - siminovInitialized()}: It is triggered whenever Siminov is initialized. 

					\begin{center}
						\colorbox{grey}{
						\parbox[t]{.8\linewidth}{
							\fontsize{11pt}{11pt}\selectfont % The first argument for fontsize is the font size of the text and the second is the line spacing - you may need to play with these for your particular title
							\vspace*{0.1cm} % Space between the start of the title and the top of the grey box
		
							\hfill \textbf{Note} \\
								This doesnot gets triggered when Siminov is first time initialized, instead of this firstTimeSiminovInitialized APIwill be triggered.

							\vspace*{0.0cm} % Space between the end of the title and the bottom of the grey box
						}
					}

					\end{center}

		
			\item \small \textbf{Siminov Stopped - siminovStopped()}: It is triggered when Siminov is shutdown.


		\end{enumerate}



		\textbf{i.ISiminov  Native Event Handler}
			
			\lstinputlisting[language=Java]{Resources/isiminov_native_event_handler_example.txt}
		
		\textbf{ii. ISiminov Web Event Handler}

			\lstinputlisting[language=Java]{Resources/isiminov_web_event_handler_example.txt}




\section{IDatabaseEvents} It provide API's related to database operations.


		\par
		\textbf{Example:} IDatabaseEvents Notifier
			\lstinputlisting[language=Java]{Resources/idatabase_events_notifier.txt}


		\begin{enumerate}

			\item \small \textbf{Database Created - databaseCreated(DatabaseDescriptor)}: It is triggered when database is created based on schema defined in DatabaseDescriptor.si.xml file. This API provides DatabaseDescriptor object for which database is created.

			\item \small \textbf{Database Dropped - databaseDropped(DatabaseDescriptor)}: It is triggered when database is dropped. This API provides DatabaseDescriptor object for which database is dropped.


			\item \small \textbf{Table Created - tableCreated(DatabaseMappingDescriptor)}: It is triggered when a table is created in database. This API provides DatabaseMappingDescriptor object which describes table structure.


			\item \small \textbf{Table Dropped - tableDropped(DatabaseMappingDescriptor)}: It is triggered when a table is deleted from database. This API provides DatabaseMappingDescriptor object for which table is dropped.


			\item \small \textbf{Index Created - indexCreated(DatabaseMappingDescriptor, Index)}: It is triggered when a index is created on table. This API provides DatabaseMappingDescriptor and Index object which defines table and index structure.

			
			\item \small \textbf{Index Dropped - indexDropped(DatabaseMappingDescriptor, Index)}: It is triggered when a index is dropped from table. This API provides DatabaseMappingDescriptor and Index object which defines table and index for which index is dropped.

		\end{enumerate}



		\textbf{i.IDatabase Native Event Handler}
			
			\lstinputlisting[language=Java]{Resources/idatabase_native_event_handler_example.txt}
		
		\textbf{ii. IDatabase Web Event Handler}

			\lstinputlisting[language=Java]{Resources/idatabase_web_event_handler_example.txt}
